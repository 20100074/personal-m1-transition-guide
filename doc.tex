% Compilation: lualatex doc.tex && lualatex doc.tex
\documentclass[b5j,titlepage]{ltjsarticle}

\usepackage{listings}
\usepackage{jvlisting}
\usepackage{datetime2}
\usepackage{hyperref}

\title{自分による自分のための \\ ARM64環境への移行ガイド}
\author{20100074}
\date{最終更新: \DTMnow}

\pagestyle{headings}

\lstset{
	numbers=left,
	tabsize=2,
	xleftmargin=0pt,
	framexleftmargin=0pt,
	frame=single,
	breaklines=true,
	basicstyle=\ttfamily,
	commentstyle=\ttfamily
}

\setcounter{tocdepth}{3}

\begin{document}
	\maketitle

	\tableofcontents

	\clearpage

	\section{初期設定}
		% 普通にセットアップする。
		% softwareupdateで最新版まで更新
		電源を入れたら,ガイドに従って設定をする。

		デスクトップが表示されたら,\texttt{softwareupdate}コマンドで最新版のmacOSまでアップデートする。

		\begin{lstlisting}[language=bash]
	sudo softwareupdate --list --all --verbose
	sudo softwareupdate --install --all --agree-to-license \ 
		--restart --verbose
		\end{lstlisting}

	\clearpage

	\section{Homebrewとソフトウェア群のインストール}
		\subsection{Homebrewのインストール}
			% xcode-select
			\begin{lstlisting}
	sudo xcode-select --install
			\end{lstlisting}
			% Rosetta 2
			\begin{lstlisting}
	sudo softwareupdate --install-rosetta --agree-to-license
			\end{lstlisting}
			% Homebrew -> /opt/Homebrew
			\begin{lstlisting}
	/bin/bash -c "$(curl -fsSL https://raw.githubusercontent.com/Homebrew/install/HEAD/install.sh)"
			\end{lstlisting}
		
		\subsection{ソフトウェア群の自動インストール}
			Intel\ Macと同様,\texttt{brew\ install}と\texttt{brew\ install\ --cask}を用いてソフトウェアをインストールしてゆく。
			% このあたりでjlistings環境を呼び出す

			その前に,Homebrewをアップデートしておく。
				\begin{lstlisting}
	brew update
				\end{lstlisting}

			\subsubsection{コマンドラインツールのインストール}
				\begin{lstlisting}
	brew install \ 
		fish vim aria2 bat bash cmake coreutils exa exiftool pv \ 
		ffmpeg gawk gcal gnu-sed gnu-tar grep inetutils iperf jq \ 
		less ldid make mp4v2 neofetch neovim nkf parallel git curl \ 
		pinentry-mac smartmontools wget xz youtube-dl \ 
		terminal-notifier rsync gnupg pandoc tmux #wine-crossover
				\end{lstlisting}

			\subsubsection{GUIツールのインストール}
				\begin{lstlisting}

	brew install --cask \ 
		1password alacritty amethyst google-chrome iterm2 \ # mactex 
		keepingyouawake skim protonmail-bridge vlc xquartz keka

\end{lstlisting}

		\subsection{手動でインストールするソフトウェア}
			\subsubsection{プリンタドライバ}
				2021年2月現在,家のプリンタはNECのColor\ MultiWriter\ 5800Cである。このプリンタをM1\ Macで動作させるためには,ドライバを手動でインストールする必要がある。

				M1対応ドライバは,\protect\url{https://jpn.nec.com/printer/laser/download/driver/color/5800c.html}からダウンロード\footnote{直リン:\ \protect\url{http://search.casnavi.nec.co.jp/printer/module/3835/ncmacopsprnstd2012bm107cml.dmg}}できる。

			\subsubsection{Retroactive}
				Appleは,macOS\ 10.15\ Catalinaにおいて,iTunesを廃止した。それゆえ,Catalina以降のmacOSではiTunesは動作しない。

				Retroactiveを使用すれば,このような状況の中でも,従来のバージョンのiTunesを最新のmacOSで動作させることができる。

				Retroactiveは,\protect\url{https://github.com/cormiertyshawn895/Retroactive/releases}からダウンロード\footnote{直リン:\ \protect\url{https://github.com/cormiertyshawn895/Retroactive/releases/download/1.9/Retroactive.1.9.zip}}できる。

			\subsubsection{MacTeX\ 2020\ Universal\ Binary}
				Homebrew\ caskで配布されているMacTeXはx86\_64向けバイナリであるため,こちら(\protect\url{http://www.tug.org/mactex/MacTeX-2020-Universal.pkg})を使用する。

			\subsubsection{Scansnap\ Home\ (Optional)}
				あまり頻繁には使用しないのでインストールしなくても良いかな,と思っているので,Optional。

				必要ならばここ(\protect\url{http://scansnap.fujitsu.com/jp/dl/mac-1100-ix500.html})からダウンロードできるようだが,Apple\ Silicon搭載MacではUSB経由でのスキャンに不具合があるらしい\footnote{更新予定はあるものの,具体的な更新日は未定のよう。}。

	\clearpage

	\section{OSの設定}
		\subsection{CUIでの設定}
			\subsubsection{\texttt{sudo}をTouchID対応にする}
				以下のコマンドを使用すれば,\texttt{sudo}をTouchIDで認証できる。

				もっとも,SSHでM1\ Macにログインした時に,\texttt{sudo}が使えなくなる可能性がある。

				\begin{lstlisting}[language=bash]
	# Desktopにcpした上でsudoeditで変更したほうがよいかも。
	# cp /etc/pam.d/sudo ~/Desktop/
	echo 'auth sufficient pam_tid.so' | sudo tee -a /etc/pam.d/sudo
				\end{lstlisting}

			\subsubsection{ホストネームの設定}
				% hostnameはAmethyst.localにするつもり。
				以下のコマンドを実行することにより,M1\ Macのホスト名(Bonjourで名前解決が可能)が\texttt{Amethyst.local}となる。

				\begin{lstlisting}[language=bash]
	sudo hostname Amethyst.local
				\end{lstlisting}

			\subsubsection{\texttt{fish}をデフォルトのシェルにする}
				以下のコマンドを実行することにより,\texttt{fish}がデフォルトのシェルとなる。

				\begin{lstlisting}[language=bash]
	# which fish
	# echo '/opt/homebrew/bin/fish' | sudo tee -a /etc/shells
	chsh -s $(which fish) -u $(whoami)
				\end{lstlisting}

			\subsubsection{\texttt{PATH}環境変数の編集}
				\begin{lstlisting}
	export PATH=/opt/homebrew/bin:$PATH
	# LaTeXについては以下のどちらか
	export PATH=/usr/local/texlive/2020/bin/arm64-darwin:$PATH
	export PATH=/usr/local/texlive/2020/bin/custom:$PATH
				\end{lstlisting}

		\subsection{GUIでの設定}
			\subsubsection{修飾キーの設定}
				以下のようにキーを入れ替える\footnote{\protect\url{file:///System/Library/PreferencePanes/Keyboard.prefPane}}。

				\begin{itemize}
					\item{\texttt{Caps\ Lock}を\texttt{Command}へ}
					\item{\texttt{Control}を\texttt{Caps\ Lock}へ}
					\item{\texttt{Command}を\texttt{Control}へ}
				\end{itemize}

			\subsubsection{SSHサーバーの有効化}
				「システム環境設定」内の「共有」で「リモートログイン」を有効化する\footnote{\protect\url{file:///System/Library/PreferencePanes/SharingPref.prefPane}}。

			\subsubsection{1920*1200の擬似解像度を有効化}
				詳しくは\protect\url{https://fantastic-works.com/archives/3154}を参照。

			\subsubsection{トラックパッドの設定}
				右クリックやダブルクリックなどの設定を「システム環境設定」で行う\footnote{\protect\url{file:///System/Library/PreferencePanes/Trackpad.prefPane}}。

			\subsubsection{バッテリーの残量表示}
				「システム環境設定」で行う\footnote{\protect\url{file:///System/Library/PreferencePanes/Battery.prefPane}}。

			\subsubsection{Time\ MachineにHDDを追加}
				「システム環境設定」で行う\footnote{\protect\url{file:///System/Library/PreferencePanes/TimeMachine.prefPane}}。

	\clearpage

	\section{ソフトウェアの設定}
		\subsection{CUIでの設定}
			\subsubsection{設定ファイルが存在するもの}
				設定ファイルが存在するソフトウェアの設定については,\texttt{scripts}フォルダの中にある\texttt{copy-preferences.sh}を実行すれば終了する。

				なお,以下に主要な設定ファイルの内容を示す(GitHub公開版PDFでは削除されている)ので,手入力をしてもよい。

				% \lstinputlisting[caption=\~/.vimrc]{~/.vimrc}
				% \lstinputlisting[caption=\~/.config/nvim/init.vim]{~/.config/nvim/init.vim}
				% \lstinputlisting[caption=\~/.config/fish/config.fish]{~/.config/fish/config.fish}
				% \lstinputlisting[caption=\~/.ssh/config]{~/.ssh/config}

				% \begin{lstlisting}[caption=\~/.gnupg/]
				%\end{lstlisting}

			\subsubsection{設定ファイルが存在しないもの}
				特に記述することはない。
				
		\subsection{GUIでの設定}
			思い当たることがあれば改めてここに記述する。

	\clearpage
	
	\section{Intel\ Macからのデータコピー}
		\subsection{IP\ over\ Thunderbolt}
			移行元のMacBook\ Pro(2016)と移行先のMacBook\ Pro(2020,\ M1)はどちらもThunderbolt\ 3をサポートしている。

			Apple\ Thunderbolt\ 3ケーブルをこの2台のMacBook\ Proに接続すると,IP\ over\ Thunderboltによるファイル転送が使用できるようになる\footnote{詳しくは\protect\url{https://support.apple.com/ja-jp/guide/mac-help/mchld53dd2f5/mac}を参照すること。DHCPによる設定がうまくいかない場合,システム環境設定のネットワークパネル(\protect\url{file:///System/Library/PreferencePanes/Network.prefPane})内にある「Thunderboltブリッジ」の設定を手動で変更する必要がある。}。

		\subsection{\texttt{rsync}}
			\texttt{rsync}を使用して,以下のディレクトリを\texttt{Amethyst.local}にコピーする(もっとも,上の2つについては,HDDに保存してあるtarアーカイブを利用したほうが速いかもしれない)。

			\begin{itemize}
				\item{\texttt{\$HOME/iTunes}}
				\item{\texttt{\$HOME/Dictionaries}}
				\item{\texttt{\$HOME/Pictures/Wallpaper}}
			\end{itemize}
	
			\begin{lstlisting}[language=bash]
	# これで本当にうまくいくかはわからない
	# ~/.ssh/config にホストを追加しておく必要がありそう
	rsync -avP6 Jasmine.local:~/iTunes ~/
	rsync -avP6 Jasmine.local:~/Dictionaries ~/
	rsync -avP6 Jasmine.local:~/Pictures/Wallpaper ~/Pictures
			\end{lstlisting}

	\clearpage

\end{document}
